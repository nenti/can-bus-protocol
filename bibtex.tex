@INPROCEEDINGS{5283124, 
author={Wampler, D. and Huirong Fu and Ye Zhu}, 
booktitle={Information Assurance and Security, 2009. IAS '09. Fifth International Conference on}, title={Security Threats and Countermeasures for Intra-vehicle Networks}, 
year={2009}, 
month={aug.}, 
volume={2}, 
number={}, 
pages={153 -157}, 
abstract={Controller area network (CAN) is the leading serial bus system for embedded control. More than two billion CAN nodes have been sold since the protocol's development in the early 1980s. CAN is a mainstream network and was internationally standardized (ISO 11898-1) in 1993. This paper describes an approach to implementing security services on top of a higher level controller area network (CAN) protocol, in particular, CANopen. Since the CAN network is an open, unsecured network, every node has access to all data on the bus. A system which produces and consumes sensitive data is not well suited for this environment. Therefore, a general-purpose security solution is needed which will allow secure nodes access to the basic security services such as authentication, integrity, and confidentiality.}, 
keywords={CANopen;ISO 11898-1;authentication service;confidentiality service;controller area network protocol;embedded control;integrity service;intra-vehicle networks;security threats;serial bus system;ISO standards;automobiles;controller area networks;embedded systems;message authentication;protocols;system buses;}, 
doi={10.1109/IAS.2009.350}, 
ISSN={},}

@ARTICLE{1632017, 
author={Barranco, M. and Proenza, J. and Rodriguez-Navas, G. and Almeida, L.}, 
journal={Industrial Informatics, IEEE Transactions on}, title={An active star topology for improving fault confinement in CAN networks}, 
year={2006}, 
month={may}, 
volume={2}, 
number={2}, 
pages={ 78 - 85}, 
abstract={ The controller area network (CAN) is a field bus that is nowadays widespread in distributed embedded systems due to its electrical robustness, low price, and deterministic access delay. However, its use in safety-critical applications has been controversial due to dependability limitations, such as those arising from its bus topology. In particular, in a CAN bus, there are multiple components such that if any of them is faulty, a general failure of the communication system may happen. In this paper, we propose a design for an active star topology called CANcentrate. Our design solves the limitations indicated above by means of an active hub, which prevents error propagation from any of its ports to the others. Due to the specific characteristics of this hub, CANcentrate is fully compatible with existing CAN controllers. This paper compares bus and star topologies, analyzes related work, describes the CANcentrate basics, paying special attention to the mechanisms used for detecting faulty ports, and finally describes the implementation and test of a CANcentrate prototype.}, 
keywords={ CAN bus; CAN networks; CANcentrate; active star topology; communication system fault diagnosis; communication system fault tolerance; controller area network; distributed embedded systems; field buses; controller area networks; fault diagnosis; fault tolerant computing; field buses; protocols; telecommunication network topology;}, 
doi={10.1109/TII.2006.875505}, 
ISSN={1551-3203},}
