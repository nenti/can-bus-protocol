\nocite{can2spec}
\nocite{canbook1}
\nocite{vector:2012:online}

\section{Introduction}
Modern automotive systems include many different hardware elements that interact
with other parts of the car. The Controller Area Network (CAN) is a bus system
developed by Bosch which meets the requirements of most parts used in a modern
car. The racing team of the university Paderborn uses the CAN bus system to
control the hardware elements of their racecar. To optmize the car's performance
on the track we are implementing a monitoring system which is able to monitor
and manipulate the CAN data from the bus. To understand the stream of data
transfered over the bus this paper will look into details of the CAN protocol.
The goal for the readers of this paper is to gain general knowledge about the
protocol so that they are able to program the interface between the CAN bus and
the monitoring GUI.

\section{Fieldbus for automotive systems}
Automotive systems include an increasing number of embedded electronic devices
to meet consumers requirements like entertainment, safety or drive assistance.
Powerfull IT innovations for vehicles are possible by interconnection and
communication between various electronic components. To efficiently implement
this interconnection, fieldbus systems were invented. 
	
	\cite{can-florian-rathgeber} 
	\subsection{Historical Placement of CAN} 
	Increasing cabeling costs and weight for the peer-to-peer
	connections between various controllers in cars introduced the demand for the
	fieldbus technology. In 1983 the Rober Bosch GmbH started the development of an
	in-vehicle network. The result of this research was officially introduced 1986
	as the CAN protocol. Only one year later the first hardware controllers were
	released by Intel and Philips Semiconductors. Even though CAN was developed for
	cars early applications were in the machine manufactoring. After Bosch released
	the CAN specification v2.0 1990 the ISO standard 11898 for CAN was published.
	Around the same time Mercedes Benz started using CAN in some of their
	series-production vehicles. Many other car manufacturors today use CAN or its
	later ancestors for their vehicle controller communication.
	
	\subsection{Requirements for automotive bus systems}
	Besides the reduced cable cost bus protocols solve more communication related
	problems. Error correction is one of the feature added to the CAN protocol by
	detecting transfer errors and propagating the missmatch to all connected CAN
	nodes. Often various different subsystems use similar sensors and therefor a
	bussystem enables the reuse of sensor data to save costs. In vehicles there are
	different systems with changing requirements on the bussystems. For example an
	airbag has hard real time requirements to ensure that the bag is filled before
	the head hits the dashboard. Whereas the entertainment system of a car needs a
	high throughput but is not required to have error correction in the bus. To address
	those requirements SAE described four classes of automotive requirements on
	bussystems.\cite{sae-classes} 
	
	\paragraph{CLASS A}
	This class describes low-end communication without many requirements on the
	bus. The goal for a bus that serves this class is to be cheap, provide a low
	bandwidth and no extra features like error correction. Mostly this kind of
	bus system is used for direct communication between sensors and actuators. The
	market leading bussystem used for this type of communication is called LIN
	(local interconnect networks)
	\paragraph{CLASS B}
	Systems that fit this class are using non-critical communication with a fairly
	low speed requirement of approximately 125Kb/sec. A good example for a system
	like that is the car body comfort system. Whereas it is not important that the
	windscreen whiper starts within a few milliseconds after you hit the button,
	error detection and correction is fairly important in those kind of systems to
	ensure a comfortable, deterministic user environment. The lowspeed CAN bus
	would be a typical bus system for this class because it provides excelent error
	correction and enough speed to meet the requirements.
	\paragraph{CLASS C}
	Class C covers expensive systems with high bandwidth, fault-tolerance and
	real-time requirements. Typical transmission speed is around 500 Kb/sec but
	modern systems like FlexRay or the upcoming CAN-FD (flexible
	datarate)\cite{canfd-conference} provide higher speed while maintaining or even
	extending the maximum buslength over the typical 40 meter. An example for a
	Class C system would be the engine control system.
	
	For our racing car, whose IT-infrastructure is not as cost-sensitive and
	complex as the one used in modern personal vehicles we only use one bus system.
	There are various different CAN nodes in place that meet all previously
	mentioned SAE classes. Using the high speed CAN bus solution ISO
	11898-2\cite{iso11898-2}, as our bussystem was a logical choice since it is
	fairly fault-tolerant and provides enough bandwidth for all connected devices.
	
	After this short introduction into fieldbus systems and the CAN system, the
	next chapter gives insight into how the CAN nodes communicate with each other
	following the CAN protocol. And how the CAN protocol provides the discussed
	feature of fault tolerance.
	

\section{CAN protocol properties} 
Bussystems implement protocols that define the language of communication that
each bus node follows for data transmission or reception.
While many network protocols are very complex, defining encapsulation for each
OSI layer.
Field bus systems like CAN only need to implement the physical layer and the
data link layer of the OSI model\autoref{fig:can_layers} . The physical layer
defines how the bytes are physically transmitted over the wire and therefor is very important when it
comes to maximum bus length and transmission speed. The data link layer
describes the actual language spoken, by defining the encapsulation and how the
data frame is used to achieve the features provieded by the bus. 

\begin{figure}[htb] \centering
	\includegraphics[width=1\textwidth]{content/pictures/OSI_CAN}
	\caption{OSI Layers on the CAN system layers}
	\label{fig:can_layers}
\end{figure}

This chapter will give a detailed insight into the data link layer of the CAN
protocol and will not describe the physical layer implementation. This is
because our project group is only interested in the CAN bus protocol as it's
seen by the programmer and we assume that the physical layer works independently
and doesn't need to be manipulated to achieve our goals. However if you are
interested in the physical layer you can read the CAN Specification
2.0\cite{can2spec} by Bosch or directly go through the ISO standards ISO 11898-2\cite{iso11898-2} or ISO
11898-3\cite{iso11898-3}.

In order to understand the protocol implementation for the CAN features one
thing you need to know about the physical layer is that it uses event or message
based communication so messages are send on demand and not in regular time
intervals. However there is an implementation of CAN called
TT-CAN\cite{iso11898-4} which implements time triggered communication to meet
the requirements of certain scheduling techniques used in real-time operating systems.
Also note that the CAN bus uses a two-wired-AND implementation, so it compares
the electric tension of one wire to the other to detect transmitted bits. In the
frame description we talk about dominant and recessive bits where dominant bits
represent a logical '0' and recessive bits repressent a logical '1'.

Before going into implementation details the following table outlines the
properties of the CAN protocols to give you an overview of the content in the
upcoming subchapters:

\begin{center}
    \begin{tabular}{ | p{35em} |}
    \hline
    \rowcolor{lightgray} Properties of the CAN protocol \\ \hline
    prioritization of messages \\ \hline
    guarantee of latency times \\ \hline
    configuration flexibility \\ \hline
    multicast reception with time synchronization \\ \hline
    system wide data consistency \\ \hline
    multimaster \\ \hline
    error detection and signalling \\ \hline
    automatic retransmission of corrupted messages as soon as the bus is idle
    again \\ \hline 
    distinction between temporary errors and permanent
    failures of nodes and autonomous switching off of defect nodes\\
    \hline
    \end{tabular}
\end{center}

\subsection{Encapsulation}
Message framing is one of the main aspects of reading and understanding the
bitstream coming from the CAN bus to our monitoring system. This section will
outline the different message formats and types and go through their frame data
structure in detail.
	\subsubsection{Frame Formats}
	To avoid overhead in smaller networks CAN implements two frame formats by
	varying the length of the identifier. Standard frames include an 11 bit
	identifier whereas extended frames include a 29 bit identifier. In our racing
	car we only use the extended frame format.
	\subsubsection{Frame Types}
	The implementation of the previous shown properties of the CAN protocol
	requires different frame types that are send to: communicate data (Data Frame),
	to request data (Remote Frame), to propagate a transmission error (Error Frame) or
	to delay further transmissions because of an overloaded node (Overload Frame).
	
	\paragraph{Data Frame} is the main transmission frame which is used to
	send data over the bus. One data frame consists of a maximum of 8 bytes of data which are
	encapsulated like in \autoref{fig:extended_data_frame}
	
	\begin{figure}[htb] \centering
		\includegraphics[width=1\textwidth]{content/pictures/data_frame.png}
		\caption{Detailed view of the Extended Data Frame \cite{vector:2012:online}}
		\label{fig:extended_data_frame}
	\end{figure}
	
	Ths Start of Frame (SOF) consists of at least one dominant bit which signals
	that the bus is idle. This is important to ensure synchronization.
	
	The Arbitration Field defines the identifier of the message. For extended data
	frames the identifier consists of a 11 bit base id where the first transmitted
	bit is the most significant one. Then there are two control bits of which the
	IDE bit is there to identify the use of extended format by transmitting a
	recessive bit. Followed by the extended it with 18 bits. The base id is very
	important when it comes to arbitration because it defines the base priority of
	the message.
	Also included into the Arbitration Field, as its last bit, following the
	extended id is the so called RTR bit which distinguishes the remote frame from the data
	frame. In a data frame the RTR bit is dominant.
	
	The Control Field consists of 6 bits that serve the main purpose of of defining
	the DLC (Data Length Code). In standard frames the first 2 bits are used for
	the SRR and IDE bits, but since in extended frames those control bits are
	defined in the Arbitration Field, the first two bits of the Control Field are
	unused reserved bits which should be dominant but need to be accepted by
	receivers either way. The DLC defines the number of bytes that will be
	transmitted in the Data Field, where DLC3\autoref{fig:extended_data_frame} is
	the most significant bit.
	
	The Data Field then finally includes the data that need to be submitted. Its
	length is variable from 0 to 64 bit. 
	
	It is then followed by the CRC Field which includes the CRC sequence generated
	over the entire frame up to the end of the Data Field starting with the SOF
	bit. The CRC field is finished by a delimiting recessive bit. 
	
	After the CRC Field the last two bits of the data frame form the ACK Field
	which allows the receivers of the message to confirm the
	successfull transmission of the data frame.
	
	Before the bus is cleared a flag sequence of seven recessive bits is send which
	is called EOF sequence.
	
	The very low maximum length of 8 byte per frame provides a predictable
	communication delay, which is important for real time communication. Anyhow
	to improve throughput a new CAN specification is currently under development
	which is called FD CAN. FD stands for flexible datarate. So to increase
	throughput and decrease frame overhead the 8 byte limit will be extended
	depending on the bus length. 
	 
	\paragraph{Remote Frame} allows actors to request data from another CAN node.
	The frame looks basically the same as the Data Frame. To signal a Remote
	Frame, the RTR bit of the Arbitration Field is set to recessive. The Remote
	Frame is only a request, there is no acutal data sent, so it leaves out the
    Data Field while still maintaining the Control Field to signal the expected
	length of the corresponding Data Frame.
	The identifier of the Remote Frame is set to the identifier of the requested
	message and doesn't include any information about the source of the request.
	
	\paragraph{Error Frame} is used to propagat an error that occured during data
	transmission. It consists of two fields. The first field contains the Error
	Flag and the second field is the Error Delimiter. 
	There are two different Error Flags, the Active Error Flag and the Passive
	Error Flag. An ``error active'' station that detects an error condition starts
	transmission of 6 consecutive dominant bits which themself form the Active
	Error Flag. This violates bit stuffing and therefor provokes all nodes to
	switch to an error state and therefor allows cancelation of data transmissions
	in progress. So the first field of the Error Frame is actually called superposition of Error Flags
	because it consists of the Error Flags of all nodes and can therefor become up
	to 12 bits long.
	The Passive Error Flag consists of 6 consecutive recessive bits but also
	accepts 6 dominant bits from an Active Error Frame as a successful Error Flag
	proceeding to the Error Delimiter.
	The Error Delimiter consists of eight recessive bits that start when the
	first recessive bit is detected on the bus, after successfully transmitting the
	Error Flag.
	
    
    \paragraph{Overload Frame} looks exactly the same as an Active Error Frame.
    The only difference is that it has different start conditions and doesn't
    increase the error counter. There are three conditions that lead to the transmission of the
    overload frame. While an Error Frame is transmitted during the
    transmission of a Data- or a Remote Frame, an Overload Frame is only
    started after the last bit of a delimiter. They are for example used to postpone any
    further transmission on the bus due to some internal condition, for example
    an overloaded processing unit.
    The bus communication can be postponed by transmitting a snail of Overload Frames
    each starting at the end of the delimiter of its preceding Overload Frame.
    To prevent a permanent blocking of the bus, at most two consecutive Overload
    Frames are allowed to delay further communication. The other two conditions for
    Overload Frames are unexpected dominant bits in the Interframe Space or the delimiter of an Error Frame.
    
        
	\subsection{Bus Arbitration}
	Arbitration is an important part of a bus system which has several nodes
	connected that compete for access on the shared bus. The CAN bus uses
	priority based arbitration where the message identifier also represents the
	message priority. The arbitration on a CAN bus is based on the principle of
	silently failing a lost arbitration and therefor it is possible to arbitrate
	competing transmissions on the fly. 
	
	On an idle bus any node can start transmitting its data frame. If two nodes
	start transmission at the same bit they are competing against each other for
	bus access.
	They keep sending their frame while monitoring the bus as a listener. If a dominant
	bit (0) is measured although a recessive bit was send, the node looses
	arbitration and turns itself from a transmitter into a receiver node. This is
	important because the message of the winning node might be intended for the
	competing node that lost the arbitration. The bitwise arbitration process is
	core element of the bus access system used on the CAN bus which is called:
	``Carrier Sense, Multiple Access / Collision Avoidance'' (CSMA/CA).
	
	\begin{figure}[htb] \centering
		\includegraphics[width=1\textwidth]{content/pictures/arbitration.png}
		\caption{Example for bus arbitration of three simultanious sending CAN nodes}
		\label{fig:arbitration}
	\end{figure}
	
	Competing communication nodes have the problem of higher latency due to the
	waiting time for other transmission. According to Lawrenz (1999)
	\cite{canbook-lawrenz} for low bus loads (smaller 10\%) this delay can be
	neglected (p.
	33).
	For applications with real time requirements, for example in closed loop
	control systems, the arbitration system needs to be taken into consideration to ensure
	meeting the latency requirements of those systems. Therefor the priorities of
	realtime systems need to be set to a reasonable value to ensure that the
	highest possible delay for the message transmission is smaller than the
	slowest tolerable reaction time of the system.
	
	\subsection{Message Filtering}
	Filtering messages is a key feature of a bus system that
	multicasts all messages to all bus nodes. The CAN bus doesn't support switching
	or routing, therefor filtering messages at the receiving nodes is used to
	identify relevant messages. The identifier that is specified in the Arbitration
	Field of the Data Frame ``does not indicate the destination of the message, but
	the meaning of the data'' \cite{can2spec}. This means, that bus nodes can send
	various different messages with various different identifiers and the bus doesn't know
	where they are from, nor where they are headed to. 
	
	Each receiver filters the
	relevant data and sends it to the application layer. Many CAN applications use
	message filtering based on the data received from the object
	layer \autoref{fig:extended_data_frame}.
	Filtering messages on the application layer will also be important for the CAN
	monitoring system we are developing. To prevent runtime errors we are required
	to validate the data format of the received messages before sending it to the
	userinterface.
	
		
	\section{Error Management}
	CAN is a very reliable bus system because it runs synchronous and
	guarantees data consistency over all nodes. This is possible because of the
	error detection and management procedures embedded into the bus protocol. This
	chapter will give insight in what types of errors there are, how the
	countermeasures work and how the CAN bus deals with error prone devices.
	
	\subsection{Fault Confinement}
	The bus systems relies on the proper functionality of each device connected. In
	case of an error prone device on the bus, the communication between all devices
	can be substantially affected. When a valid transmission is detected as
	invalid because of an internal failure of a CAN node, it keeps being
	invalidated by the active error packet sent by this node. To prevent this from
	happening, the fault confinement part of the CAN protocol defines three states
	a bus device can have:
	\subparagraph{Error active} units are the default participants in the bus
	communications and are allowed to send Active Error Flags when an error has
	been detected. 
	\subparagraph{Error passive} units are not allowed to send an Active Error Flag
	but keep participating in the normal bus communication. They loose the
	priviledge to block ongoing transmissions on an error detection. In case of a
	detected error, the error passive CAN node sends a Passive Error Flag which
	only consists of recessive bits. A passive Error Frame is only detected by
	other devices when it's propagated by the transmitter of the current message.
	\subparagraph{Bus off} units are not allowed to send any traffice to the bus
	anymore. The output drivers are switched off and the application layer might
	need to execute further steps if a failure of the device is fatal for the
	entire system.
	
	To detect defect CAN nodes, the fault confinment process implements two counts
	that need to be maintained by every node: Transmit Error Count and Receive
	Error Count. Basically each error that occures has a penaulty that increases
	the error counter of each node accordingly. There are mechanisms in place that
	detect if the node itself was the first one to detect the error (high penaulty
	8) or if it collectively detected the error with all other nodes (low penaulty
	1). There are 12 rules in place that define the conditions and exceptions for
	penaulties and state transitions in detail \cite{can2spec}. As an example for
	the development of the error counter on a can bus look at
	\autoref{fig:error_counter}.
	
	\begin{figure}[htb] \centering 
		\includegraphics[width=1\textwidth]{content/pictures/error_counter.png}
		\caption{Example for CAN error counters (p. 97, Lawrenz,
		1999)\cite{canbook-lawrenz}}
		\label{fig:error_counter}
	\end{figure}
	
	When the Transmit Error Count or the Receive Error Count exceeds 127, the node
	becomes error passive until both counts are decreased below 128 again. Because
	an error passive node can't do any damage to the bus communication when
	receiving data anymore, in order to become bus off, only the Transmit Error
	Count is taken into consideration. When the Transmit Error Counter exceeds the
	mark of 255, the node is turned bus off. The only possibility to leave this
	state again is on an idle bus, where it's affordable to check if `bus off' nodes
	might work again. In detail the `bus off' node monitors the bus and counts the
	occurence of 11 consecutive recessive bits. If this happened 128 times, so after
	1408 bus idle bits the node sets both counters to zero and returns bus active
	again. Penaulties can count up to a critical level very fast because one error
	can count for up to eight penaulty points. On the other side the counter is only
	decreased by one penaulty point for each successfull data transmission. The CAN
	protocol considers error counts above 96 as critical and suggests to monitor
	the bus for that level to prevent critical failures. The observation of the
	error counts might also be a valuable suggestion for our CAN monitoring system
	to detect communication problems on the bus via the user interface.
	
	After this introduction to the error management system of the CAN bus the next
	chapter outlines the different error conditions on the bus. Each error
	condition is propagated over the bus by sending an Error Frame and the error
	count is increased as discussed in this chapter.
	
	\subsection{Error Types}
	The CAN bus protocol strictly defines frame formats and includes
	countermeasures like the cyclic redundancy check (CRC) or 
	bit stuffing. Therefor the protocol defines different error types that occure
	on error conditions during the communication. Now that we know how the ``Error Frame''
	looks like, this chapter looks at the events that trigger the nodes to change
	to an error state and intercept ongoing communication.
	
	\subsubsection{Bit Error} 
	A Bit Error is detected by the transmitter monitoring the bus
	while sending data on it to detect a missmatch. This method is also used in
	the arbitration process. During the
	arbitration, the Bit Error detection is disabled so the transmission can
	silently fail. Anywhere else in the Data Frame or any other frame type a
	Bit Error will be detected, when the transmitter sends a recessive bit but
	monitors a dominant bit on the bus. The error will be propagated over the bus
	by sending an Error Frame aborting the ongoing transmission. There are
	other exceptions where a Bit Error is suppressed to allow certain processes
	to work:
	\begin{itemize}
	  \item The Ack Field is designed to have receivers send a dominant bit on
	a received recessive bit from the transmitter to confirm the successfull
	reception of the entire packet.
	  \item Monitoring a
	  dominant bit on transmission of a Passive Error Flag, is also not
	  interpreted as a Bit Error.
	\end{itemize} 
	
	\subsubsection{Stuff Error}
	The main frame segments of the Data Frame are coded by a method called bit
	stuffing. This means that whenever the transmitter monitors five consecutive
	bits of the same value on the bus, it automatically sends a complementary bit.
	A Stuff Error is triggered when a 6th consecutive bit of the same value is
	detected in field that is supposed to be coded with bit stuffing. That way an
	Active Error Flag that consists of six consecutive dominant bits provokes a
	Stuff Error on all other nodes listening to the bus, by intentionally not
	following the coding rules of bit stuffing.
	
	\subsubsection{CRC Error}
	\label{subsec:crc_check}
	The Cyclic Redundancy Check (CRC) is a commonly used error-detection method to
	detect random bit changes in raw data. Automotive bus systems are used in a
	rough environment side by side with a lot of different technologies and
	therefor need to prevent errors that often occure in their operational area.
	For example caused by noise in the transmission channel. The CAN bus implements
	a CRC check for the entire Data Frame up to the CRC Field. This ensures
	integrity of the transmitted data. If the CRC Sequence calculated by the
	receiver, over the transmitted data stream, doesn't match the CRC Sequence in
	the received CRC Frame, an CRC Error is propagated and the transmission of the
	packet has to restart. In order to meet the bit requirement of the implemented
	CRC configuration which should check less than 127 bits at a time, only the
	destuffed bitstream is checked.

\section{Bus Security Concerns} 
Our project goal to connect a CAN bus with a wireless monitoring system
introduces a new threat to our car. Whereas the CAN bus used to be a closed
system which therefor was very hard to attack, we open up our bus
system to an uncontrollable audience by sending and receiving data via wireless
technology. Even though available countermeasures for the wireless
transmission is available, undesired wireless interaction with our CAN bus can
not be ruled out. This chapter will therefor outline upcoming threats for embedded CAN
communication and present proposals for countermeasures.

\subsection{Denial of Service}
As previously discussed the CAN specification uses bitwise arbitration for its
access system CSMA/CA. The arbitration forces CAN nodes to back of from their
transmission when they detect a higher prioritized node transmitting a message
at the same time. This simple but very effective system together with the
possibility of external access from a potential attacker to a CAN node
introduces a high risk potential. The attacker can send senseless Data Frames
with the highest priority on the highest possible rate and therefor jam the
entire bus communication. This is also why Marko Wolf et al. (2004) call this
type of attack the ``Jamming Attack''. For this attack scenario I prefered the
more common known name ``Denial of Service'' following the work of Wampler and
Fu (2009) \cite{security-wampler-huirong} because it follows
the ``Denial of Service'' principles of spamming a system so it breaks down,
in this case so the bus is jammed.

For our monitoring system this is a very real threat because it might be planned
for the future to allow tweaking CAN devices via the monitoring system for
testing purposes. Therefor the interface must allow wireless devices to send CAN
messages to the bus. Furthermore it must be possibe to set the identifiers
of a message to be able to manipulate CAN devices on the bus. This would allow
attackers to use this interface for a ``Denial of Service'' attack.

Possible countermeasures against this threat would be to disassemble the device
after the testing process. An on/off switch might be forgotten to turn off after
testing and therefor should be considered as an insufficient approach.
Furthermore it would be possible to include a basic firewall into the wireless interface device
that only allows low priority messages to be send by this device. This would
allow defining security relevant messages under a high priority level that is
filtered on the interface device and therefor not allowed to be manipulated by
it.

\section{Conclusion}
The CAN bus is a very reliable bus system because of its long historic
background and the sophisticated devices available on the market. Furthermore it
is still under development to keep up with the latest technologies while
maintaining its high standards of reliability.

%  Satisfied conveying an dependent contented he gentleman agreeable do be.
% Warrant private blushes removed an in equally totally if. Delivered dejection
% necessary objection do mr prevailed. Mr feeling do chiefly cordial in do.
% Water timed folly right aware if oh truth. Imprudence attachment him his for
% sympathize. Large above be to means. Dashwood do provided stronger is. But
% discretion frequently sir the she instrument unaffected admiration everything.
% Advice me cousin an spring of needed.
%  You can see this in \autoref{fig:mypicture}.
%  \begin{figure}[htb] \centering
% \includegraphics[width=0.7\textwidth]{content/pictures/mypicture}
% \caption{description of the picture} \label{fig:mypicture} \end{figure}  Tell
% use paid law ever yet new. Meant to learn of vexed if style allow he there.
% Tiled man stand tears ten joy there terms any widen. Procuring continued
% suspicion its ten. Pursuit brother are had fifteen distant has. Early had add
% equal china quiet visit. Appear an manner as no limits either praise in. In in
% written on charmed justice is amiable farther besides. Law insensible
% middletons unsatiable for apartments boy delightful unreserved.
% How promotion excellent curiosity yet attempted happiness. Gay prosperous
% impression had conviction. For every delay death ask style. Me mean able my by
% in they. Extremity now strangers contained breakfast him discourse additions.
% Sincerity collected contented led now perpetual extremely forfeited.
% Woody equal ask saw sir weeks aware decay. Entrance prospect removing we
% packages strictly is no smallest he. For hopes may chief get hours day rooms.
% Oh no turned behind polite piqued enough at. Forbade few through inquiry
% blushes you. Cousin no itself eldest it in dinner latter missed no. Boisterous
% estimating interested collecting get conviction friendship say boy. Him mrs
% shy article smiling respect opinion excited. Welcomed humoured rejoiced
% peculiar to in an.
% Improve ashamed married expense bed her comfort pursuit mrs. Four time took ye
% your as fail lady. Up greatest am exertion or marianne. Shy occasional
% terminated insensible and inhabiting gay. So know do fond to half on. Now who
% promise was justice new winding. In finished on he speaking suitable advanced
% if. Boy happiness sportsmen say prevailed offending concealed nor was
% provision. Provided so as doubtful on striking required. Waiting we to compass
% assured.
% Six started far placing saw respect females old. Civilly why how end viewing
% attempt related enquire visitor.
%  Example for referencing: You can see this in \cite{example}.
%  \subsection{SubSection B} Satisfied conveying an dependent contented he
% gentleman agreeable do be. Warrant private blushes removed an in equally
% totally if. Delivered dejection necessary objection do mr prevailed. Mr
% feeling do chiefly cordial in do. Water timed folly right aware if oh truth.
% Imprudence attachment him his for sympathize. Large above be to means.
% Dashwood do provided stronger is. But discretion frequently sir the she
% instrument unaffected admiration everything.
% Advice me cousin an spring of needed.
%  In \autoref{lst:example-java-code} you can see an example of including Java
% source code into the paper.
%  % format java source code \lstinputlisting[language=Java,caption=sample in
% Java,label=lst:example-java-code]{content/code/example.java}  Tell use paid
% law ever yet new. Meant to learn of vexed if style allow he there. Tiled man
% stand tears ten joy there terms any widen. Procuring continued suspicion its
% ten. Pursuit brother are had fifteen distant has. Early had add equal china
% quiet visit. Appear an manner as no limits either praise in. In in written on
% charmed justice is amiable farther besides. Law insensible middletons
% unsatiable for apartments boy delightful unreserved.
% How promotion excellent curiosity yet attempted happiness.
%  \subsection{SubSection C}  Gay prosperous impression had conviction. For
% every delay death ask style.
% Me mean able my by in they. Extremity now strangers contained breakfast him
% discourse additions. Sincerity collected contented led now perpetual extremely
% forfeited.
% Woody equal ask saw sir weeks aware decay. Entrance prospect removing we
% packages strictly is no smallest he. For hopes may chief get hours day rooms.
% Oh no turned behind polite piqued enough at. Forbade few through inquiry
% blushes you. Cousin no itself eldest it in dinner latter missed no. Boisterous
% estimating interested collecting get conviction friendship say boy. Him mrs
% shy article smiling respect opinion excited. Welcomed humoured rejoiced
% peculiar to in an. Improve ashamed married expense bed her comfort pursuit
% mrs. Four time took ye your as fail lady. Up greatest am exertion or marianne.
% Shy occasional terminated insensible and inhabiting gay. So know do fond to
% half on. Now who promise was justice new winding. In finished on he speaking
% suitable advanced if. Boy happiness sportsmen say prevailed offending
% concealed nor was provision. Provided so as doubtful on striking required.
% Waiting we to compass assured.
% Six started far placing saw respect females old. Civilly why how end viewing
% attempt related enquire visitor. Man particular insensible celebrated
% conviction stimulated principles day. Sure fail or in said west. Right my
% front it wound cause fully am sorry if. She jointure goodness interest
% debating did outweigh. Is time from them full my gone in went. Of no
% introduced am literature excellence mr stimulated contrasted increasing. Age
% sold some full like rich new. Amounted repeated as believed in confined
% juvenile.
% Can curiosity may end shameless explained.
%  % format C source code  \begin{figure}[htb] \lstinputlisting[language=C,
% caption=sample in C]{content/code/example.c} \end{figure}  True high on said
% mr on come. An do mr design at little myself wholly entire though. Attended of
% on stronger or mr pleasure. Rich four like real yet west get. Felicity in
% dwelling to drawings. His pleasure new steepest for reserved formerly disposed
% jennings.
% Much did had call new drew that kept. Limits expect wonder law she. Now has
% you views woman noisy match money rooms. To up remark it eldest length oh
% passed. Off because yet mistake feeling has men. Consulted disposing to
% moonlight ye extremity. Engage piqued in on coming.
% Increasing impression interested expression he my at.
%  \begin{center} \begin{tabular}{ | l | l | l | p{5.5cm} |} \hline
% \rowcolor{lightgray} Day & Min Temp & Max Temp & Summary \\ \hline Monday &
% $11^\circ$C & $22^\circ$C & A clear day with lots of sunshine.
% However, the strong breeze will bring down the temperatures. \\ \hline Tuesday
% & $9^\circ$C & $19^\circ$C & Cloudy with rain, across many northern regions.
% Clear spells across most of Scotland and Nor\-thern Ireland, but rain reaching
% the far northwest. \\ \hline Wednesday & $10^\circ$C & $21^\circ$C & Rain will
% still linger for the morning.
% Conditions will improve by early afternoon. \\
% \hline \end{tabular} \end{center}  Respect invited request charmed me warrant
% to. Expect no pretty as do though so genius afraid cousin. Girl when of ye
% snug poor draw. Mistake totally of in chiefly. Justice visitor him entered
% for. Continue delicate as unlocked entirely mr relation diverted in. Known not
% end fully being style house. An whom down kept lain name so at easy.
% Started his hearted any civilly. So me by marianne admitted speaking. Men bred
% fine call ask. Cease one miles truth day above seven. Suspicion sportsmen
% provision suffering mrs saw engrossed something. Snug soon he on plan in be
% dine some.