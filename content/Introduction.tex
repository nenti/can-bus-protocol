\nocite{can2spec}
\nocite{canbook1}
\nocite{vector:2012:online}

\section{Introduction}
Modern automotive systems include many different hardware elements that interact
with other parts of the car. The Controller Area Network (CAN) is a bus system
developed by Bosch which meets the requirements of most parts used in a modern
car. The racing team of the university Paderborn uses the CAN bus system to
control the hardware elements of their racecar. To optmize the car's performance
on the track we are implementing a monitoring system which is able to monitor
and manipulate the CAN data from the bus. To understand the stream of data
transfered over the bus this paper will look into details of the CAN protocol.
The goal for the readers of this paper is to gain general knowledge about the
protocol so that they are able to program the interface between the CAN bus and
the monitoring GUI.

\section{Fieldbus for automotive systems}
Automotive systems include an increasing number of embedded electronic devices
to meet consumers requirements like entertainment, safety or drive assistance.
Powerfull IT innovations for vehicles are possible by interconnection and
communication between various electronic components. To efficiently implement
this interconnection, fieldbus systems were invented. 
	
	\cite{can-florian-rathgeber} 
	\subsection{Historical Placement of CAN} 
	Increasing cabeling costs and weight for the peer-to-peer
	connections between various controllers in cars introduced the demand for the
	fieldbus technology. In 1983 the Rober Bosch GmbH started the development of an
	in-vehicle network. The result of this research was officially introduced 1986
	as the CAN protocol. Only one year later the first hardware controllers were
	released by Intel and Philips Semiconductors. Even though CAN was developed for
	cars early applications were in the machine manufactoring. After Bosch released
	the CAN specification v2.0 1990 the ISO standard 11898 for CAN was published.
	Around the same time Mercedes Benz started using CAN in some of their
	series-production vehicles. Many other car manufacturors today use CAN or its
	later ancestors for their vehicle controller communication.
	
	\subsection{Requirements for automotive bus systems}
	Besides the reduced cable cost bus protocols solve more communication related
	problems. Error correction is one of the feature added to the CAN protocol by
	detecting transfer errors and propagating the missmatch to all connected CAN
	nodes. Often various different subsystems use similar sensors and therefor a
	bussystem enables the reuse of sensor data to save costs. In vehicles there are
	different systems with changing requirements on the bussystems. For example an
	airbag has hard real time requirements to ensure that the bag is filled before
	the head hits the dashboard. Whereas the entertainment system of a car needs a
	high throughput but is not required to have error correction in the bus. To address
	those requirements SAE described four classes of automotive requirements on
	bussystems.\cite{sae-classes} 
	
	\paragraph{CLASS A}
	This class describes low-end communication without many requirements on the
	bus. The goal for a bus that serves this class is to be cheap, provide a low
	bandwidth and no extra features like error correction. Mostly this kind of
	bus system is used for direct communication between sensors and actuators. The
	market leading bussystem used for this type of communication is called LIN
	(local interconnect networks)
	\paragraph{CLASS B}
	Systems that fit this class are using non-critical communication with a fairly
	low speed requirement of approximately 125Kb/sec. A good example for a system
	like that is the car body comfort system. Whereas it is not important that the
	windscreen whiper starts within a few milliseconds after you hit the button,
	error detection and correction is fairly important in those kind of systems to
	ensure a comfortable, deterministic user environment. The lowspeed CAN bus
	would be a typical bus system for this class because it provides excelent error
	correction and enough speed to meet the requirements.
	\paragraph{CLASS C}
	Class C covers expensive systems with high bandwidth, fault-tolerance and
	real-time requirements. Typical transmission speed is around 500 Kb/sec but
	modern systems like FlexRay or the upcoming CAN-FD (flexible
	datarate)\cite{canfd-conference} provide higher speed while maintaining or even
	extending the maximum buslength over the typical 40 meter. An example for a
	Class C system would be the engine control system.
	
	For our racing car, whose IT-infrastructure is not as cost-sensitive and
	complex as the one used in modern personal vehicles we only use one bus system.
	There are various different CAN nodes in place that meet all previously
	mentioned SAE classes. Using the high speed CAN bus solution ISO
	11898-2\cite{iso11898-2}, as our bussystem was a logical choice since it is
	fairly fault-tolerant and provides enough bandwidth for all connected devices.
	
	After this short introduction into fieldbus systems and the CAN system, the
	next chapter gives insight into how the CAN nodes communicate with each other
	following the CAN protocol. And how the CAN protocol provides the discussed
	feature of fault tolerance.
	

\section{CAN bus protocol}
Bussystems implement a protocol that defines the language of communication that
each bus node needs to speak in order to understand each other. Where many
network protocols are very complex, defining encapsulation for each OSI layer.
Field bus systems like CAN only need to implement the physical layer and the
data link layer of the OSI model\autoref{fig:can_layers} . The physical layer
defines how the bytes are physically transmitted over the wire and therefor is very important when it
comes to maximum bus length and transmission speed. The data link layer
describes the actual language spoken, by defining the encapsulation and how the
data frame is used to achieve the features provieded by the bus. 

\begin{figure}[htb] \centering
	\includegraphics[width=1\textwidth]{content/pictures/OSI_CAN}
	\caption{OSI Layers on the CAN system layers}
	\label{fig:can_layers}
\end{figure}

This chapter will give a detailed insight into the data link layer of the CAN
protocol and will not describe the physical layer implementation. This is
because out project group is only interested in the CAN bus protocol as it's
seen by the programmer and we assume that the physical layer works independently
and doesn't need to be manipulated to achiebe our goals. However if you are
interested in the physical layer you can read the CAN Specification
2.0\cite{can2spec} by Bosch or directly go through the ISO standards ISO 11898-2\cite{iso11898-2} or ISO
11898-3\cite{iso11898-3}.

In order to understand the protocol implementation for the CAN features one
thing you need to know about the physical layer is that it uses event or message
based communication so messages are send on demand and not in regular time
intervals. However there is an implementation of CAN called
TT-CAN\cite{iso11898-4} which implements time triggered communication to meet
the requirements of certain scheduling techniques used in real-time operating systems.
Also note that the CAN bus uses two wired AND implementation, so it compares
the electric tension of one wire to the other to detect transmitted bits. In the
frame description we talk about dominant and recessive bits where dominant bits
represent a logical '0' and recessive bits repressent a logical '1'.

Before going into implementation details the following table outlines the
properties of the CAN protocols to give you an overview of the content in the
upcoming subchapters:

\begin{center}
    \begin{tabular}{ | l | p{5.5cm} |}
    \hline
    \rowcolor{lightgray} Properties of the CAN protocol \\ \hline
    prioritization of messages & jkljkljkl \\ \hline
    guarantee of latency times & iopiopiop \\ \hline
    configuration flexibility & iopiopiop \\ \hline
    multicast reception with time synchronization & iopiopiop \\ \hline
    system wide data consistency & iopiopiop \\ \hline
    multimaster & iopiopiop \\ \hline
    error detection and signalling & iopiopiop \\ \hline
    automatic retransmission of corrupted messages as soon as the bus is idle
    again & ioioio \\ \hline 
    distinction between temporary errors and permanent
    failures of nodes and autonomous switching off of defect nodes & afsdfasd \\
    \hline
    \end{tabular}
\end{center}

\subsection{Encapsulation}
Message framing is one of the main aspects of reading and understanding the
bitstream coming from the CAN bus to our monitoring system. This section will
outline the different message formats and types and go through their frame data
structure in detail.
	\subsubsection{Frame Formats}
	To avoid overhead in smaller networks CAN implements two frame formats by
	varying the length of the identifier. Standard frames include an 11 bit
	identifier whereas extended frames include a 29 bit identifier. In our racing
	car we only use the extended frame format.
	\subsubsection{Frame Types}
	Implementation of the previous shown properties of the CAN protocol requires
	different frame types that are send to communicate data (Data Frame), to
	request data (Remote Frame), to propagate a transmission error (Error Frame) or
	to delay further transmissions because of an overloaded node (Overload Frame).
	\paragraph{Extended Data Frame}
	Data frame is the main transmission frame which is used to send data over the
	bus. One data frame consists of a maximum of 8 bytes of data which are
	encapsulated like in \autoref{fig:extended_data_frame}
	
	\begin{figure}[htb] \centering
		\includegraphics[width=1\textwidth]{content/pictures/EDF}
		\caption{Detailed view of the Extended Data Frame}
		\label{fig:extended_data_frame}
	\end{figure}
	
	Ths Start of Frame (SOF) consists of at least one dominant bit which signals
	that the bus is idle. This is important to ensure synchronization.
	
	The Arbitration Field defines the identifier of the message. For extended data
	frames the identifier consists of a 11 bit base id where the first transmitted
	bit is the most significant one. Then there are two control bits of which the
	IDE bit is there to identify the use of extended format by transmitting a
	recessive bit. Followed by the extended it with 18 bits. The base id is very
	important when it comes to arbitration because it defines the base priority of
	the message.
	Also included into the Arbitration Field as the last bit following the extended
	id is the so called RTR bit which distinguishes the remote frame from the data
	frame. In a data frame the RTR bit is dominant.
	
	The Control Field consists of 6 bits that serve the main purpose of of defining
	the DLC (Data Length Code). In standard frames the first 2 bits are used for
	the SRR and IDE bits, but since in extended frames those control bits are
	defined in the Arbitration Field, the first two bits of the Control Field are
	unused reserved bits which should be dominant but need to be accepted by
	receivers either way. The DLC defines the number of bytes that will be
	transmitted in the Data Field, where DLC3\autoref{fig:extended_data_frame} is
	the most significant bit.
	
	-Flexible Datarate
	 and therefor
	provides a predictable communication delay, which is important for real time
	operating systems.
	\paragraph{Remote Frame}
Requesting data from another node.
	\paragraph{Error Frame}
Propagation of an error occured in a transmission.
\subsection{Bus Arbitration}
Arbitration is an important part of a bus system which has several nodes
connected that want to communicate over the shared bus. The CAN bus uses
priority based arbitration which will be described in this section. I will also
include a discussion about real time requirements and the problem of suppressed
low priority nodes.
\subsection{Message Filtering}
In a bus system which uses broadcast and not packet-switching based data
transmission, filtering messages and identifying the correct receiver is a key
feature which will be described in this section.
\subsection{Error Management and Synchronisation}
CAN is a very reliable bus system because it runs synchronous and uses various
error detection and management procedures which will be described in this
section
	\paragraph{CRC Check}
In this subsection I will outline the CRC check.
	\paragraph{Bit Stuffing}
Bit Stuffing is used to prevent synchronisation errors.

\section{Bus Security Concerns}
Our project goal to connect a CAN bus with a wireless monitoring system
introduces a new threat to our car. Whereas the CAN bus used to be a closed
system which therefor was very hard to attack, we open up our bus
system to an uncontrollable audience by sending and receiving data via wireless
technology. Even though available countermeasures for the wireless
transmission is used, undesired wireless interaction with our CAN bus can not be
ruled out. This chapter will therefor outline upcoming threats for embedded CAN
communication and present proposals for countermeasures.

\section{Conclusion}
The CAN bus is a very reliable bus system because of its long historic
background and the sophisticated devices available on the market. Furthermore it
is still under development to keep up with the latest technologies while
maintaining its high standards of reliability.

%  Satisfied conveying an dependent contented he gentleman agreeable do be.
% Warrant private blushes removed an in equally totally if. Delivered dejection
% necessary objection do mr prevailed. Mr feeling do chiefly cordial in do.
% Water timed folly right aware if oh truth. Imprudence attachment him his for
% sympathize. Large above be to means. Dashwood do provided stronger is. But
% discretion frequently sir the she instrument unaffected admiration everything.
% Advice me cousin an spring of needed.
%  You can see this in \autoref{fig:mypicture}.
%  \begin{figure}[htb] \centering
% \includegraphics[width=0.7\textwidth]{content/pictures/mypicture}
% \caption{description of the picture} \label{fig:mypicture} \end{figure}  Tell
% use paid law ever yet new. Meant to learn of vexed if style allow he there.
% Tiled man stand tears ten joy there terms any widen. Procuring continued
% suspicion its ten. Pursuit brother are had fifteen distant has. Early had add
% equal china quiet visit. Appear an manner as no limits either praise in. In in
% written on charmed justice is amiable farther besides. Law insensible
% middletons unsatiable for apartments boy delightful unreserved.
% How promotion excellent curiosity yet attempted happiness. Gay prosperous
% impression had conviction. For every delay death ask style. Me mean able my by
% in they. Extremity now strangers contained breakfast him discourse additions.
% Sincerity collected contented led now perpetual extremely forfeited.
% Woody equal ask saw sir weeks aware decay. Entrance prospect removing we
% packages strictly is no smallest he. For hopes may chief get hours day rooms.
% Oh no turned behind polite piqued enough at. Forbade few through inquiry
% blushes you. Cousin no itself eldest it in dinner latter missed no. Boisterous
% estimating interested collecting get conviction friendship say boy. Him mrs
% shy article smiling respect opinion excited. Welcomed humoured rejoiced
% peculiar to in an.
% Improve ashamed married expense bed her comfort pursuit mrs. Four time took ye
% your as fail lady. Up greatest am exertion or marianne. Shy occasional
% terminated insensible and inhabiting gay. So know do fond to half on. Now who
% promise was justice new winding. In finished on he speaking suitable advanced
% if. Boy happiness sportsmen say prevailed offending concealed nor was
% provision. Provided so as doubtful on striking required. Waiting we to compass
% assured.
% Six started far placing saw respect females old. Civilly why how end viewing
% attempt related enquire visitor.
%  Example for referencing: You can see this in \cite{example}.
%  \subsection{SubSection B} Satisfied conveying an dependent contented he
% gentleman agreeable do be. Warrant private blushes removed an in equally
% totally if. Delivered dejection necessary objection do mr prevailed. Mr
% feeling do chiefly cordial in do. Water timed folly right aware if oh truth.
% Imprudence attachment him his for sympathize. Large above be to means.
% Dashwood do provided stronger is. But discretion frequently sir the she
% instrument unaffected admiration everything.
% Advice me cousin an spring of needed.
%  In \autoref{lst:example-java-code} you can see an example of including Java
% source code into the paper.
%  % format java source code \lstinputlisting[language=Java,caption=sample in
% Java,label=lst:example-java-code]{content/code/example.java}  Tell use paid
% law ever yet new. Meant to learn of vexed if style allow he there. Tiled man
% stand tears ten joy there terms any widen. Procuring continued suspicion its
% ten. Pursuit brother are had fifteen distant has. Early had add equal china
% quiet visit. Appear an manner as no limits either praise in. In in written on
% charmed justice is amiable farther besides. Law insensible middletons
% unsatiable for apartments boy delightful unreserved.
% How promotion excellent curiosity yet attempted happiness.
%  \subsection{SubSection C}  Gay prosperous impression had conviction. For
% every delay death ask style.
% Me mean able my by in they. Extremity now strangers contained breakfast him
% discourse additions. Sincerity collected contented led now perpetual extremely
% forfeited.
% Woody equal ask saw sir weeks aware decay. Entrance prospect removing we
% packages strictly is no smallest he. For hopes may chief get hours day rooms.
% Oh no turned behind polite piqued enough at. Forbade few through inquiry
% blushes you. Cousin no itself eldest it in dinner latter missed no. Boisterous
% estimating interested collecting get conviction friendship say boy. Him mrs
% shy article smiling respect opinion excited. Welcomed humoured rejoiced
% peculiar to in an. Improve ashamed married expense bed her comfort pursuit
% mrs. Four time took ye your as fail lady. Up greatest am exertion or marianne.
% Shy occasional terminated insensible and inhabiting gay. So know do fond to
% half on. Now who promise was justice new winding. In finished on he speaking
% suitable advanced if. Boy happiness sportsmen say prevailed offending
% concealed nor was provision. Provided so as doubtful on striking required.
% Waiting we to compass assured.
% Six started far placing saw respect females old. Civilly why how end viewing
% attempt related enquire visitor. Man particular insensible celebrated
% conviction stimulated principles day. Sure fail or in said west. Right my
% front it wound cause fully am sorry if. She jointure goodness interest
% debating did outweigh. Is time from them full my gone in went. Of no
% introduced am literature excellence mr stimulated contrasted increasing. Age
% sold some full like rich new. Amounted repeated as believed in confined
% juvenile.
% Can curiosity may end shameless explained.
%  % format C source code  \begin{figure}[htb] \lstinputlisting[language=C,
% caption=sample in C]{content/code/example.c} \end{figure}  True high on said
% mr on come. An do mr design at little myself wholly entire though. Attended of
% on stronger or mr pleasure. Rich four like real yet west get. Felicity in
% dwelling to drawings. His pleasure new steepest for reserved formerly disposed
% jennings.
% Much did had call new drew that kept. Limits expect wonder law she. Now has
% you views woman noisy match money rooms. To up remark it eldest length oh
% passed. Off because yet mistake feeling has men. Consulted disposing to
% moonlight ye extremity. Engage piqued in on coming.
% Increasing impression interested expression he my at.
%  \begin{center} \begin{tabular}{ | l | l | l | p{5.5cm} |} \hline
% \rowcolor{lightgray} Day & Min Temp & Max Temp & Summary \\ \hline Monday &
% $11^\circ$C & $22^\circ$C & A clear day with lots of sunshine.
% However, the strong breeze will bring down the temperatures. \\ \hline Tuesday
% & $9^\circ$C & $19^\circ$C & Cloudy with rain, across many northern regions.
% Clear spells across most of Scotland and Nor\-thern Ireland, but rain reaching
% the far northwest. \\ \hline Wednesday & $10^\circ$C & $21^\circ$C & Rain will
% still linger for the morning.
% Conditions will improve by early afternoon. \\
% \hline \end{tabular} \end{center}  Respect invited request charmed me warrant
% to. Expect no pretty as do though so genius afraid cousin. Girl when of ye
% snug poor draw. Mistake totally of in chiefly. Justice visitor him entered
% for. Continue delicate as unlocked entirely mr relation diverted in. Known not
% end fully being style house. An whom down kept lain name so at easy.
% Started his hearted any civilly. So me by marianne admitted speaking. Men bred
% fine call ask. Cease one miles truth day above seven. Suspicion sportsmen
% provision suffering mrs saw engrossed something. Snug soon he on plan in be
% dine some.